\documentclass[a4paper, 14pt]{report}

\setcounter{secnumdepth}{3}
\setcounter{tocdepth}{3}

% \usepackage{biblatex}
\usepackage[vietnamese=nohyphenation]{hyphsubst}
\usepackage[english, vietnamese]{babel}
\usepackage{indentfirst} % indent first paragraph
\usepackage{geometry} % set page margin

\usepackage{tikz} %vẽ bìa

\usepackage{enumerate}
\usepackage{amsmath}
\usepackage{amsfonts}
\usepackage{amssymb}
\usepackage{color}


\usepackage{pgfplots}


\usepackage{amsthm}
\usepackage[unicode]{hyperref}
\usepackage{fontspec}
\usepackage{graphicx}
\usepackage{enumitem}
\usepackage{a4wide}
\usepackage{epsfig}
\usepackage{latexsym}
\usepackage{array}
\usepackage{hhline}

\usepackage[normalem]{ulem}
\usepackage[makeroom]{cancel}
\usepackage{longtable}
\usepackage{amscd}
\usepackage{diagbox}
\usepackage{booktabs}
\usepackage{alltt}
\usepackage[framemethod=tikz]{mdframed}
\usepackage{subcaption}
\usepackage{caption}
\usepackage{listings}
\usepackage{xcolor}
\usepackage{lipsum}
\usepackage{setspace}

\usepackage{multicol}
\usepackage{indentfirst}
\usepackage{float}
\usepackage{titlesec}
\usepackage[nottoc]{tocbibind}
\usepackage[autostyle]{csquotes}



\usetikzlibrary{decorations}
\usetikzlibrary{decorations.pathreplacing}
\usetikzlibrary{decorations.pathreplacing,calligraphy}
\usetikzlibrary{arrows.meta}
\usetikzlibrary{quotes}
\usetikzlibrary{intersections}
\usetikzlibrary{calc}
\usetikzlibrary{arrows}

\setstretch{1.3}

\pgfplotsset{compat=1.15}

\setlength{\baselineskip}{6pt}

\theoremstyle{definition}


\renewcommand\thechapter{\Roman{chapter}}
\renewcommand\thesection{\arabic{section}}
\renewcommand\thesubsection{\thesection.\arabic{subsection}}
\renewcommand*{\thesubsubsection}{\thesubsection.\arabic{subsubsection}}

\newtheorem{theorem}{Định lý}[section]
\newtheorem{acknowledgement}[theorem]{Acknowledgement}
\newtheorem{algorithm}[theorem]{Algorithm}
\newtheorem{axiom}[theorem]{Axiom}
\newtheorem{case}[theorem]{Case}
\newtheorem{claim}[theorem]{Claim}
\newtheorem{conclusion}[theorem]{Conclusion}
\newtheorem{condition}[theorem]{Condition}
\newtheorem{conjecture}[theorem]{Conjecture}
\newtheorem{corollary}[theorem]{Corollary}
\newtheorem{criterion}[theorem]{Criterion}
\newtheorem{definition}{Định nghĩa}
\newtheorem{example}{Ví dụ}
\newtheorem{exercise}[theorem]{Exercise}
\newtheorem{lemma}[theorem]{Lemma}
\newtheorem{notation}[theorem]{Notation}
\newtheorem{problem}[theorem]{Problem}
\newtheorem{proposition}[theorem]{Mệnh đề }
\newtheorem{remark}{Nhận xét}
\newtheorem{solution}[theorem]{Solution}
\newtheorem{summary}[theorem]{Summary}

\geometry{
    top=2.5cm,
    bottom=3cm,
    left=3.5cm,
    right=2cm
}

\setmainfont{Times New Roman}
\titleformat{\section}
  {\bfseries\fontsize{18}{20}\selectfont\rmfamily}{\thesection}{1em}{}
\titleformat{\subsection}
  {\bfseries\fontsize{16}{18}\selectfont\rmfamily}{\thesubsection}{1em}{}
\titleformat{\subsubsection}
  {\bfseries\fontsize{14}{16}\selectfont\rmfamily}{\thesubsubsection}{1em}{}

\definecolor{dkgreen}{rgb}{0,0.6,0}
\definecolor{gray}{rgb}{0.5,0.5,0.5}
\definecolor{mauve}{rgb}{0.58,0,0.82}
\everymath={\displaystyle}
\newmdenv[linecolor=gray,skipabove=\topsep,skipbelow=\topsep,
leftmargin=-5pt,rightmargin=-5pt,
innerleftmargin=5pt,innerrightmargin=5pt]{mybox}

\lstset{
  % frame=tb,
  language=Python,
  aboveskip=3mm,
  belowskip=3mm,
  showstringspaces=false,
  columns=flexible,
  basicstyle={\small\ttfamily},
  numbers=left,
  numberstyle=\tiny\color{gray},
  keywordstyle=\color{blue},
  commentstyle=\color{dkgreen},
  stringstyle=\color{mauve},
  breaklines=true,
  breakatwhitespace=true,
  tabsize=4
}


\title{Blockchain trong truy xuất nguồn gốc thực phẩm}

\author{
{
    Lê Thị Thùy Dung\thanks{Khoa Toán-Cơ-Tin học, Đại học Khoa học Tự Nhiên, }}
}
\date{}


\begin{document}
\pagestyle{empty}
% \input{Bia1.tex}
% \begin{titlepage}
    \begin{tikzpicture}[overlay,remember picture]
    \draw [line width=2pt]
        ($ (current page.north west) + (3.5cm,-2.5cm) $)
        rectangle
        ($ (current page.south east) + (-2cm,2.5cm) $);
        \draw [line width=0.5pt]
    ($ (current page.north west) + (3.6cm,-2.6cm) $)
    rectangle
    ($ (current page.south east) + (-2.1cm,2.6cm) $); 
    \end{tikzpicture}
    \begin{center}        
        {\fontsize{14}{10}\selectfont 
        ĐẠI HỌC QUỐC GIA HÀ NỘI\\
        ĐẠI HỌC KHOA HỌC TỰ NHIÊN\\
        \fontsize{13}{16}\selectfont \textbf{KHOA TOÁN - CƠ - TIN HỌC}
        }
        \end{center}
        
        \vspace*{3cm}
        
        \begin{center}
            {
                \fontsize{14}{0}\selectfont \textbf{LÊ THỊ THUỲ DUNG}
            }
        \end{center}
        \begin{center}
       
        \vspace*{3cm}
        {\fontsize{18}{24}\selectfont \textbf{ BLOLCKCHAIN TRONG TRUY XUẤT NGUỒN \\ GỐC THỰC PHẨM}}
        \end{center}
        \vspace*{2cm}
        \begin{center}
        {
        \fontsize{14}{16}\selectfont Khoá luận tốt nghiệp dành cho đại học chính quy \\
        \fontsize{14}{16}\selectfont Ngành Máy tính và Khoa học Thông tin \\
        (Chương trình đào tạo chuẩn)
        }
        \end{center}
  
        \vspace*{1cm}
        \begin{center}
            {
                \fontsize{14}{0}\selectfont \textbf{ Cán bộ hướng dẫn: PGS.TS. PHÓ ĐỨC TÀI}
            }
        \end{center}
        \vfill
         \centerline{
            \fontsize{14}{0}\selectfont \textbf{ HÀ NỘI - 2023}}
        
    \end{titlepage}
    
    
% \input{Loicamon.tex}
\newpage
\section*{Danh mục các ký hiệu và chữ viết tắt}
\thispagestyle{empty}
\fontsize{14}{16}\selectfont
\begin{table}[htbp]
    \begin{center}
      \begin{tabular*}{\linewidth}{@{\extracolsep{\fill}}|>{\centering}m{0.1\linewidth}|>{\centering\arraybackslash}m{0.2\linewidth}|>{\centering\arraybackslash}m{0.25\linewidth}|>{\centering\arraybackslash}m{0.25\linewidth}|}
        \hline
        \textbf{STT} & \textbf{Chữ viết tắt} & \textbf{Diễn giải} & \textbf{Tiếng Việt} \\
        \hline
        01 & SHA & Secure Hash Algorithm & Thuật toán băm an toàn  \\
        \hline
    
        \hline
        Value 7 & Value 8 & Value 9 & Value 3 \\
        \hline
      \end{tabular*}
    \end{center}
  \end{table}
  
% \newpage
\section*{Danh mục các hình vẽ và đồ thị}
\thispagestyle{empty}


\tableofcontents

\newpage
\pagestyle{plain}
\section*{Mở đầu}
\setcounter{page}{1}
Truy xuất nguồn gốc thực phẩm là quá trình xác định nguồn gốc và lịch sử sản xuất của một sản phẩm
thực phẩm từ giai đoạn sản xuất đến khi đến tay người tiêu dùng. Ý nghĩa của việc truy xuất nguồn 
gốc thực phẩm là giúp người tiêu dùng đảm bảo được an toàn về chất lượng và nguồn gốc của sản phẩm
mà mình sử dụng, đồng thời cũng là một phương tiện quan trọng giúp kiểm soát và ngăn chặn sự xuất
hiện của những sản phẩm thực phẩm giả mạo, không rõ nguồn gốc và an toàn.

Việc truy xuất nguồn gốc thực phẩm đang được xem là một xu hướng quan trọng trong ngành thực phẩm,
đặc biệt là trong bối cảnh ngày nay, khi mối đe dọa về an toàn thực phẩm và việc giả mạo sản phẩm 
thực phẩm ngày càng trở nên phức tạp. Việc thực hiện truy xuất nguồn gốc thực phẩm sẽ giúp tăng
cường niềm tin của người tiêu dùng đối với sản phẩm thực phẩm và giúp nâng cao chất lượng và an toàn thực phẩm trên thị trường.

Truy xuất nguồn gốc thực phẩm được thực hiện thông qua việc ghi nhận, lưu trữ và quản lý các thông
tin liên quan đến quá trình sản xuất và phân phối sản phẩm thực phẩm, bao gồm nguồn gốc, thời gian
sản xuất, địa điểm sản xuất, thông tin về các thành phần, chất lượng, độ an toàn của sản phẩm. 
Những thông tin này được ghi nhận và kiểm soát bởi các cơ quan chức năng, doanh nghiệp và các tổ 
chức liên quan trong quá trình sản xuất và phân phối sản phẩm.

Với sự phát triển của công nghệ Blockchain, việc áp dụng nó để giải quyết vấn đề này đang trở nên 
phổ biến hơn. Công nghệ Blockchain cho phép lưu trữ thông tin về nguồn gốc, vận chuyển và quản lý 
các sản phẩm thực phẩm một cách minh bạch, chính xác và an toàn.

Trong bài toán truy xuất nguồn gốc thực phẩm sử dụng công nghệ Blockchain, các thông tin về nguồn 
gốc, chất lượng và quá trình sản xuất của sản phẩm sẽ được lưu trữ trên một mạng lưới phi tập trung,
nơi mà các bên liên quan có thể truy cập và xác minh thông tin một cách dễ dàng. Điều này giúp cho 
người tiêu dùng có thể kiểm tra được nguồn gốc và chất lượng của sản phẩm mà mình mua, giảm thiểu 
nguy cơ mua phải sản phẩm giả, kém chất lượng hoặc không đảm bảo vệ sinh an toàn thực phẩm.

Ngoài ra, việc sử dụng công nghệ Blockchain trong bài toán truy xuất nguồn gốc thực phẩm cũng giúp
cho các nhà sản xuất và đại lý có thể quản lý sản phẩm một cách chính xác hơn, đảm bảo tính minh 
bạch và giảm thiểu chi phí trong quá trình quản lý hàng hóa. Từ đó, giúp cho ngành công nghiệp thực
phẩm phát triển bền vững và đáp ứng được nhu cầu của thị trường ngày càng khắt khe.

Em sẽ trình bày khoá luận này theo hướng đi từ các kiến thức bổ trợ rồi đến bài toán chính.
Các Chương I cung cấp lý thuyết và các kiến thức cần thiết cho bài toán chính đặt ra ở Chương II.
Chương III ứng dụng các kiến thức để mô tả ứng dụng thử nghiệm.
Do còn hạn chế về kiến thức nên bài khoá luận sẽ có một số phần chưa đạt kết quả như mong muốn. 
Sự nhận xét, đánh giá và đóng góp ý kiến từ hội đồng sẽ là động lực để em hoàn thiện để
đề tài tốt hơn.

\input{section1.tex}
\input{section2.tex}
\input{section3.tex}
\input{section4.tex}
\input{section5.tex}
\input{section6.tex}
\input{section7.tex}

\bibliographystyle{acm}
\bibliography{bib}

\end{document}      