\chapter{ Truy xuất nguồn gốc thực phẩm dựa trên công nghệ Blockchain}

\section{Truy xuất nguồn gốc thực phẩm}
\label{Chapter2}
\subsection{Khái niệm truy xuất nguồn gốc thực phẩm}
Truy xuất nguồn gốc thực phẩm là quá trình xác định nguồn gốc và lịch sử sản xuất
của một sản phẩm thực phẩm từ giai đoạn sản xuất đến khi đến tay người tiêu dùng. Ý
nghĩa của việc truy xuất nguồn gốc thực phẩm là giúp người tiêu dùng đảm bảo được
an toàn về chất lượng và nguồn gốc của sản phẩm mà mình sử dụng, đồng thời cũng là
một phương tiện quan trọng giúp kiểm soát và ngăn chặn sự xuất hiện của những sản
phẩm thực phẩm giả mạo, không rõ nguồn gốc và an toàn.

Việc truy xuất nguồn gốc thực phẩm đang được xem là một xu hướng quan trọng
trong ngành thực phẩm, đặc biệt là trong bối cảnh ngày nay, khi mối đe dọa về an toàn
thực phẩm và việc giả mạo sản phẩm thực phẩm ngày càng trở nên phức tạp. Việc thực
hiện truy xuất nguồn gốc thực phẩm sẽ giúp tăng cường niềm tin của người tiêu dùng
đối với sản phẩm thực phẩm và giúp nâng cao chất lượng và an toàn thực phẩm trên thị
trường.

Truy xuất nguồn gốc thực phẩm được thực hiện thông qua việc ghi nhận, lưu trữ
và quản lý các thông tin liên quan đến quá trình sản xuất và phân phối sản phẩm thực
phẩm, bao gồm nguồn gốc, thời gian sản xuất, địa điểm sản xuất, thông tin về các thành
phần, chất lượng, độ an toàn của sản phẩm. Những thông tin này được ghi nhận và kiểm
soát bởi các cơ quan chức năng, doanh nghiệp và các tổ chức liên quan trong quá trình
sản xuất và phân phối sản phẩm.

\subsection{Các mô hình truy xuất nguồn gốc hiện nay}
\subsubsection{Mô hình theo dõi dòng sản phẩm }
Mô hình theo dõi dòng sản phẩm là một mô hình 
truy xuất nguồn gốc đơn giản nhất. Trong mô hình này, mỗi sản phẩm được gắn 
nhãn hoặc mã số duy nhất để theo dõi từ khi sản xuất đến khi đến tay người 
tiêu dùng. Thông tin về nguồn gốc, quá trình sản xuất, vận chuyển và lưu trữ 
của sản phẩm được lưu trữ trong cơ sở dữ liệu và có thể được tra cứu bởi người tiêu dùng.

Mô hình này sử dụng mã QR và mã vạch để truy xuất nguồn gốc. Các mã QR và mã vạch 
được in trên sản phẩm để cho phép người tiêu dùng quét và truy xuất thông tin về 
nguồn gốc, quá trình sản xuất và vận chuyển của sản phẩm.

Mô hình này khá đơn giản, dễ triển khai và chi phí thấp. Nó cũng giúp tăng cường 
tính minh bạch và đáng tin cậy của thông tin về nguồn gốc sản phẩm.
\subsubsection{Mô hình truy xuất nguồn gốc dựa vào IoT }
Mô hình truy xuất nguồn gốc dựa vào IoT (Internet of Things) là một mô hình truy 
xuất nguồn gốc sử dụng công nghệ IoT để thu thập và chia sẻ thông tin về nguồn gốc, 
quá trình sản xuất, vận chuyển và lưu trữ của sản phẩm.

Trong mô hình này, các cảm biến IoT được gắn trên sản phẩm để thu thập thông tin 
về vị trí, nhiệt độ, độ ẩm và các thông số khác liên quan đến quá trình sản xuất 
và vận chuyển sản phẩm. Các dữ liệu này được lưu trữ và chia sẻ trên nền tảng 
đám mây, giúp người tiêu dùng có thể truy xuất thông tin về nguồn gốc sản phẩm 
bằng cách quét mã QR hoặc mã vạch trên sản phẩm.

\subsubsection{Các vấn đề của các hệ thống truy xuất nguồn gốc hiện có}

Các mô hình truy xuất nguồn gốc hiện nay đang có một số hạn chế cần phải khắc phục. 
Các mô hình này sử dụng nguồn dữ liệu từ các cơ sở dữ liệu của các tổ chức, doanh
nghiệp, việc đảm bảo thông tin là chính xác không được xác nhận. Ngoài ra, thông 
tin được lưu trữ trên một hệ thống tập trung, do bên thứ 3 quản lý, do bên thứ 3
có thể thay đổi thông tin, làm giả mạo thông tin và mất mát thông tin trong quá
trình truyền tải. Khi bên thứ 3 bị tấn công, sửa đổi thông tin, chúng ta không 
thể kiểm soát được. Các mô hình này cũng không thể cung cấp toàn bộ chi tiết 
về nguồn gốc, quá trình sản xuất.

Vậy nên bài toán đặt ra là xây dựng một mô hình truy xuất nguồn gốc thực phẩm
sao cho thông tin lưu trữ đầy đủ các quy trình, các thông tin lưu trữ đảm bảo tính
toàn vẹn, không thể làm giả.


\section{Truy xuất nguồn gốc thực phẩm bằng công nghệ Blockchain}

Như đã trình bày ở phần \ref{Chapter2}, ta cần tìm ra một mô hình có thể đảm bảo 
tính toàn vẹn và đáng tin cậy cao, khó bị tấn công từ bên ngoài, và giúp 
cải thiện tính minh bạch và đáng tin cậy trong quá trình sản xuất và vận 
chuyển sản phẩm.  

Bài toán truy xuất nguồn gốc thực phẩm là một trong những ứng dụng tiềm năng của công nghệ 
blockchain. Bằng cách sử dụng blockchain, thông tin về nguồn gốc, vận chuyển và lưu trữ 
của các sản phẩm thực phẩm có thể được lưu trữ và truy xuất một cách an toàn và minh bạch.

\subsection{Ưu điểm mô hình truy xuất nguồn gốc thực phẩm dựa trên blockchain}
 
\begin{itemize}
    \item \textbf{Minh bạch:} Các thông tin về nguồn gốc, vận chuyển và lưu trữ của các sản phẩm 
    thực phẩm được lưu trữ và truy xuất một cách an toàn và minh bạch.
    \item \textbf{An toàn:} Các thông tin về nguồn gốc, vận chuyển và lưu trữ của các sản phẩm 
    thực phẩm được lưu trữ và truy xuất một cách an toàn và minh bạch.
    \item \textbf{Đáng tin cậy:} Các thông tin về nguồn gốc, vận chuyển và lưu trữ của các sản phẩm 
    thực phẩm được lưu trữ và truy xuất một cách an toàn và minh bạch.
\end{itemize}

\subsection{Cách tiếp cận và giải pháp}

Sử dụng nền tảng Hyperledger Fabric để xây dựng một mạng lưới blockchain phù hợp với 
mục đích truy xuất nguồn gốc thực phẩm.

\section{Mô hình hệ thống}
\subsection{Các thành phần tham gia}

Cốt lõi của hệ thống được thực hiện dưới dạng chaincode. Chaincode cung cấp các thao tác cho phép người dùng hệ thống thêm và sửa đổi 
thông tin trong blockchain một cách an toàn và có thể theo dõi được. Người sử dụng hệ thống
là các thành viên chuỗi cung ứng và các bộ phận quản lý. Các thực thể tham gia 
vào hoạt động của hệ thống là các tổ chức người dùng, trong đó mỗi người dùng được xác định 
bằng một chứng chỉ do cơ quan chứng nhận liên kết với tổ chức được xác định rõ ràng mới có thể 
tham gia vào các hoạt động của hệ thống. Trong Hyperledger Fabric, tập hợp các tổ chức tham 
gia vào các hoạt động blockchain được xác định trước. Hyperledger Fabric cho phép thêm một tổ 
chức mới hoặc xóa một tổ chức hiện có trong thời gian chạy bằng cách chuyển một loạt các 
giao dịch sang blockchain phải được đa số các tổ chức tham gia chấp thuận.

Với hệ thống truy xuất nguồn gốc thực phẩm, thành viên bao gồm: 
\begin{itemize}
    \item[-] Nhà cung cấp giống vật tư: Dữ liệu truy tìm bao gồm thông tin về nguyên liệu 
    thực phẩm nông nghiệp (ví dụ: hạt giống, thuốc trừ sâu và phân bón), giao dịch với người nông dân v.v.
    \item[-] Nông dân: Dữ liệu truy tìm bao gồm thông tin về các trang trại, quá trình canh tác,
    điều kiện thời tiết, giao dịch với các nhà sản xuất, ...
    \item[-] Nhà sản xuất: Dữ liệu truy tìm bao gồm thông tin về các nhà máy, quá trình sản xuất,
    giao dịch với nông dân và nhà phân phối,...
    \item[-] Công ty vận chuyển: Dữ liệu truy tìm bao gồm chi tiết vận chuyển, điều kiện lưu trữ, giao
    dịch với nông dân, nhà phân phối hay nhà sản xuất,...
    \item[-] Nhà phân phối: Dữ liệu truy tìm bao gồm thông tin về các cửa hàng, quá trình phân phối,
    ngày nhập hàng, hạn sử dụng, các giao dịch với nhà sản xuất,...
\end{itemize}

Tùy vào từng sản phẩm cụ thể, mô hình có thể thêm hoặc bớt các thành viên tham gia. 
\subsection{Phân tích và thiết kế hệ thống}
\subsubsection{Domain Model}

\begin{figure}[h]
    \centering
    \includegraphics[width=1\textwidth]{images/domain_model.png}
    \caption{Domain Model } 
\end{figure}

Các tổ chức (Organization) muốn tham gia vào chuỗi cung ứng sẽ đăng ký tham gia vào hệ thống. Để được xác nhận, 
tổ chức sẽ kí hợp đồng thông minh thể hiện qua chaincode. Nếu thỏa mãn hợp đồng, 
tổ chức sẽ được thêm vào và có một mã định danh ID. Mỗi tổ chức sẽ có những một vai trò nhất định, được thể hiện
thông qua RoleSet. RoleSet liên kết với các hợp đồng thông minh để xác định các hành động mà
tổ chức có thể thực hiện thông qua việc gọi hàm của chaincode.

Một tổ chức có thể đăng ký nhiều sản phẩm, loại sản phẩm khác nhau. Việc đăng ký sản phẩm
này thỏa mãn các điều kiện được thiết kế trong hợp đồng thông minh. Việc kiểm tra 
yêu cầu này được thực hiện một cách tự động thông qua các hàm của chaincode. Sản phẩm có một 
số thuộc tính như ID, type, state,...

Một sản phẩm có thể được cấu tạo từ nhiều nguyên liệu, vật tư,... Những nguyên liệu ấy
có thể đã được tồn tại trong mạng blockchain. Nếu tồn tại trong mạng blockchain, ta sẽ thêm 
thuộc tính params để thể hiện mối quan hệ. Nếu không thì ta sẽ yêu cầu giao dịch
để ghi thông tin về sản phẩm\cite{app}. 

Việc xây dựng Domain Model như này giúp việc quản lý nguồn gốc cũng như xác thực nguồn gốc 
dễ dàng hơn, như quản lý được sản phẩm được sản xuất từ lô nông sản nào, cũng từ đó truy xuất được nguồn gốc giống của sản phẩm nông sản.
\subsubsection{Chaincode}
Chaincode nó được sử dụng để xác định các đầu vào và đầu ra của các giao dịch trên blockchain. Chaincode là một chương trình được 
viết bằng một trong các ngôn ngữ lập trình.

Một số hàm tiêu biểu của chaincode trong hệ thống truy xuất nguồn gốc thực phẩm.

\begin{figure}[h]
    \centering
    \includegraphics[width=0.4\textwidth]{images/Smart_Contract.png}
    \caption{Chaincode }
\end{figure}

\begin{itemize}
    
    \item \textbf{addRoleSet():} Thêm một thành viên mới vào hệ thống. Thành viên này có thể là một tổ chức hoặc cá nhân.
    \item \textbf{addProductType():} Thêm một loại sản phẩm mới vào hệ thống. Loại sản phẩm này có thể là sản phẩm chính hoặc sản phẩm phụ.
    \item \textbf{addRule():} Thêm một quy tắc mới vào hệ thống. Quy tắc này có thể là quy tắc chính hoặc quy tắc phụ.
    \item \textbf{enableRule () và disableRule ():} Cho phép bật và tắt rule qua ruleId.
    \item \textbf{blockProductType() và unblockProductType():} cho phép chặn và mở chặn một loại sản phẩm.
    \item \textbf{requestProductRegistration():} yêu cầu đăng ký một sản phẩm mới.
    \item \textbf{acceptProductRegistration() và refuseProductRegistration():} cho phép chấp nhận hoặc từ chối đăng ký sản phẩm.
    \item \textbf{blockProduct() và unblockProduct():} cho phép chặn và mở chặn một sản phẩm.
    \item \textbf{registerBatxh():} cho phép một tổ chức đăng ký một lô sản phẩm liên kết với sản phẩm nào đó.
    \item \textbf{blockBatch() và unblockBatch():} cho phép chặn và mở chặn một lô sản phẩm.
    \item \textbf{requestBatchTransfer():} cho phép một tổ chức yêu cầu chuyển một lô sản phẩm từ một tổ chức khác.
    \item \textbf{acceptBatchTransfer() và refuseBatchTransfer():} cho phép chấp nhận hoặc từ chối chuyển lô sản phẩm.
    \item \textbf{getBatchHistory():} cho phép lấy lịch sử của một lô sản phẩm \cite{app}.
\end{itemize}
\subsubsection{Tạo mạng}
\subsection{Các bước thực hiện}
Các bước thực hiện cơ bản của hệ thống truy xuất nguồn gốc thực phẩm như sau:
\begin{itemize}
    \item[-] \textbf{Bước 1:} Các tổ chức đăng ký thành viên vào hệ thống.
        \item[-] \textbf{1.1:} Các tổ chức tạo nút thành viên. 
        \item[-] \textbf{1.2:} Các nút thành viên tham gia vào kênh.
        \item[-] \textbf{1.3:} Tổ chức tạo admin là nút khách trong hệ thống, và yêu cầu
        tham gia vào kênh. Peer node xác thực thông tin của client và cấp quyền.
    \item[-] \textbf{Bước 2:} Các tổ chức đăng ký các loại sản phẩm của riêng mình bằng cách 
    gửi yêu cầu giao dịch. Các bước thực hiện được trình bày ở \ref{subsec:luonggiaodich} chương \ref{chap:hyper}.
    \item[-] \textbf{Bước 3:} Các tổ chức lần lượt thêm các thông tin, các thông tin sẽ được
    lưu trữ trong các khối nối tiếp nhau và sắp xếp theo thời gian. Các nút thành viên sẽ lưu 
    trữ chuỗi khối này ở sổ cái.
    \item[-] \textbf{Bước 4:} Khách hàng đăng nhập vào ứng dụng và nhập mã để yêu cầu truy xuất
    thông tin sản phẩm. Nếu sản phẩm chỉ thuộc 1 kênh, không có nguyên liệu thuộc kênh
    khác thì sẽ trả về thông tin sản phẩm. Ngược lại thì sẽ truy xuất thông tin nguyên liệu
    của sản phẩm đó và trả về tất cả thông tin sản phẩm.
\end{itemize}


\section{Công nghệ sử dụng}
- Docker

- Golang

- Hyperledger Fabric

\section{Thực nghiệm}

\section{Kết luận và hướng phát triển}