\chapter{Truy xuất nguồn gốc thực phẩm}
\label{Chapter2}
\section{Khái niệm truy xuất nguồn gốc thực phẩm}
\subsection{Giới thiệu}

Truy xuất nguồn gốc thực phẩm là quá trình xác định nguồn gốc và lịch sử sản xuất
của một sản phẩm thực phẩm từ giai đoạn sản xuất đến khi đến tay người tiêu dùng. Ý
nghĩa của việc truy xuất nguồn gốc thực phẩm là giúp người tiêu dùng đảm bảo được
an toàn về chất lượng và nguồn gốc của sản phẩm mà mình sử dụng, đồng thời cũng là
một phương tiện quan trọng giúp kiểm soát và ngăn chặn sự xuất hiện của những sản
phẩm thực phẩm giả mạo, không rõ nguồn gốc và an toàn.

Việc truy xuất nguồn gốc thực phẩm đang được xem là một xu hướng quan trọng
trong ngành thực phẩm, đặc biệt là trong bối cảnh ngày nay, khi mối đe dọa về an toàn
thực phẩm và việc giả mạo sản phẩm thực phẩm ngày càng trở nên phức tạp. Việc thực
hiện truy xuất nguồn gốc thực phẩm sẽ giúp tăng cường niềm tin của người tiêu dùng
đối với sản phẩm thực phẩm và giúp nâng cao chất lượng và an toàn thực phẩm trên thị
trường.

Truy xuất nguồn gốc thực phẩm được thực hiện thông qua việc ghi nhận, lưu trữ
và quản lý các thông tin liên quan đến quá trình sản xuất và phân phối sản phẩm thực
phẩm, bao gồm nguồn gốc, thời gian sản xuất, địa điểm sản xuất, thông tin về các thành
phần, chất lượng, độ an toàn của sản phẩm. Những thông tin này được ghi nhận và kiểm
soát bởi các cơ quan chức năng, doanh nghiệp và các tổ chức liên quan trong quá trình
sản xuất và phân phối sản phẩm.

\section{Các mô hình truy xuất nguồn gốc hiện nay}
\subsection{Mô hình theo dõi dòng sản phẩm }
Mô hình theo dõi dòng sản phẩm là một mô hình 
truy xuất nguồn gốc đơn giản nhất. Trong mô hình này, mỗi sản phẩm được gắn 
nhãn hoặc mã số duy nhất để theo dõi từ khi sản xuất đến khi đến tay người 
tiêu dùng. Thông tin về nguồn gốc, quá trình sản xuất, vận chuyển và lưu trữ 
của sản phẩm được lưu trữ trong cơ sở dữ liệu và có thể được tra cứu bởi người tiêu dùng.

Mô hình này sư dụng mã QR và mã vạch để truy xuất nguồn gốc. Các mã QR và mã vạch 
được in trên sản phẩm để cho phép người tiêu dùng quét và truy xuất thông tin về 
nguồn gốc, quá trình sản xuất và vận chuyển của sản phẩm.

Mô hình này khá đơn giản, dễ triển khai và chi phí thấp. Nó cũng giúp tăng cường 
tính minh bạch và đáng tin cậy của thông tin về nguồn gốc sản phẩm.
\subsection{Mô hình truy xuất nguồn gốc dựa vào IoT }
Mô hình truy xuất nguồn gốc dựa vào IoT (Internet of Things) là một mô hình truy 
xuất nguồn gốc sử dụng công nghệ IoT để thu thập và chia sẻ thông tin về nguồn gốc, 
quá trình sản xuất, vận chuyển và lưu trữ của sản phẩm.

Trong mô hình này, các cảm biến IoT được gắn trên sản phẩm để thu thập thông tin 
về vị trí, nhiệt độ, độ ẩm và các thông số khác liên quan đến quá trình sản xuất 
và vận chuyển sản phẩm. Các dữ liệu này được lưu trữ và chia sẻ trên nền tảng 
đám mây, giúp người tiêu dùng có thể truy xuất thông tin về nguồn gốc sản phẩm 
bằng cách quét mã QR hoặc mã vạch trên sản phẩm.

\subsection{Các vấn đề của các hệ thống truy xuất nguồn gốc hiện có}

Các mô hình truy xuất nguồn gốc hiện nay đang có một số hạn chế cần phải khắc phục. 
Các mô hình này sử dụng nguồn dữ liệu từ các cơ sở dữ liệu của các tổ chức, doanh
nghiệp, việc đảm bảo thông tin là chính xác không được xác nhận. Ngoài ra, thông 
tin được lưu trữ trên một hệ thống tập trung, do bên thứ 3 quản lý, do bên thứ 3
có thể thay đổi thông tin, làm giả mạo thông tin và mất mát thông tin trong quá
trình truyền tải. Khi bên thứ 3 bị tấn công, sửa đổi thông tin, chúng ta không 
thể kiểm soát được. Các mô hình này cũng không thể cung cấp toàn bộ chi tiết 
về nguồn gốc, quá trình sản xuất.

Vậy nên bài toán đặt ra là xây dựng một mô hình truy xuất nguồn gốc thực phẩm
sao cho thông tin lưu trữ đầy đủ các quy trình, các thông tin lưu trữ đảm bảo tính
toàn vẹn, không thể làm giả.



